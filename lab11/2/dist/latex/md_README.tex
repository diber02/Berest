
\begin{DoxyEnumerate}
\item {\bfseries{Визначити}} усі парні та невід\textquotesingle{}ємні числа вхідного масиву
\end{DoxyEnumerate}

{\bfseries{Виконання роботи}}
\begin{DoxyEnumerate}
\item {\bfseries{Функціональне призначення}} Програма призначена для визначення усіх парних та невід\textquotesingle{}ємних чисел вхідного масиву
\item Написання коду програми \begin{DoxyVerb}     #define N 7   

     #include <stdlib.h> 
     int main() {

     int arr[N][N];

     int mas_res[N];


     zap_arr_rand(arr);
     calloc_mas_res(mas_res,arr);
     sort_and_fill_mas_res(arr,mas_res);


     return 0;
     }      
     int zap_arr_rand (int arr[][N]){

     for (int i = 0; i < N; i++) {
     for (int j = 0; j < N; j++) {
     *(*(arr + i)+j)= rand()%50-50;
     }
     }
     return arr;
     }
     int calloc_mas(int mas_res[]) {

     for (int i = 0; i < N; i++) {
     int *mas_res = calloc (N,sizeof(int));
     }
     }
     int sort_and_fill_mas_res(int mas_res[], int arr[][N]) {

     for (int i = 0; i < N; i++) {
     for (int j = 0; j < N; j++) {

     if ( *(*(arr + i ) + j ) % 2 == 0 && *(*(arr + i ) + j ) > 0) {
     *(mas_res + i) =  *(*(arr + i ) + j ) ;
     }
     }
     }
     for(int i=0;*(mas_res+i)!= 0; i++){
     printf("%d ", *(mas_res+i));


     }
     }
\end{DoxyVerb}

\item Компіяція програми та перевірка на правильність її роботи через nemiver
\end{DoxyEnumerate}

!\mbox{[}\mbox{]} (\href{file:///home/diana/Berest/lab08,09,10/7.1.5/doc/nemiver.png}{\texttt{ file\+:///home/diana/\+Berest/lab08,09,10/7.\+1.\+5/doc/nemiver.\+png}} ~\newline


4.Блок-\/схема

 ~\newline


\begin{DoxyAuthor}{Автор}
Berest D. 
\end{DoxyAuthor}
\begin{DoxyDate}{Дата}
20-\/dec-\/2020 
\end{DoxyDate}
\begin{DoxyVersion}{Версія}
7.\+1.\+5 
\end{DoxyVersion}
