
\begin{DoxyEnumerate}
\item {\bfseries{Продемонструвати взаємодію з користувачем шляхом використання певних функцій}}
\end{DoxyEnumerate}

{\bfseries{Виконання роботи}}
\begin{DoxyEnumerate}
\item {\bfseries{Функціональне призначення}} Програма призначена для демонстрації взаємодії з користувачем
\item Написання коду програми \begin{DoxyVerb} #include <stdio.h>
 #include <stdlib.h>
 #define SIZE 30

 void fill_array (int **array,int  N) ;//прототипи функцій

 void sort_array(int **array,int N);

 int main() 
 {
 srand(time(NULL));

 char name [SIZE];
 char group;

 printf("\n\nВведіть ім'я користувача : ");
 gets(name);
 printf ("Введіть букву своєї групи : ");
 group = getc(stdin);
 puts("Добрий день, можете почати працювати! \n");

 char s = '-';
 for (int i = 0; i < 80; i++)
 {
 putc ( s,stdout);
 }

 printf("\n\nАвтор : Берест Діана, студентка групи КІТ-120в\n\nЛабораторна робота 12\n\nТема : продемонструвати взаємодію з користувачем шляхом використання різноманітних функцій\n\n");


 for (int i = 0; i < 80; i++)
 {
     putc ( s,stdout);
 }

 int N;

 printf("Якого розміру буде ваш масив? :");
 scanf("%d",&N);

    //create a new dimensional array

 int **array = (int**)calloc(N,sizeof(int*));
 for(int i=0; i<N; i++)
 {
 array[i] = (int*)calloc(N,sizeof(int));
 }



 fill_array(array,N);
 sort_array(array,N);

 return 0;

 for (int i = 0; i < N; i++) // освобождаем память после окончания работы 
 { 
 free(array[i]);
 }
 free(array);

 }


 void fill_array (int **array,int N) 
 { 
 for (int i= 0;i<N;i++)
 {
 for (int j= 0; j <N;j++) 
 {

 *(*(array+i)+j)=rand()%10;
 /*
 printf ("arr[%d][%d] = ", i,j);
 scanf ("%d",&*(&*(array + i)+j));
 */

 }
 }
 }  


 void sort_array( int **array, int N)
 {

 int array_result[15] = {0};
 int counter = 0;
 int *p_result=&array_result[0];

 for (int i = 0; i < N; i++)
 {
 for (int j = 0; j < N; j++)
 {
 if(*(*(array+i)+j)%2==0 && *(*(array+i)+j)>0)
 { 
 *(p_result + counter + 1)=*(*(array+i)+j);
 counter++;
 }
 }
 }
 *(p_result) = counter;
 for(int i=0;*(p_result+i)!=0;i++)
 { 
 printf("%d\n",*(p_result+i));
 }
 }
\end{DoxyVerb}

\item Компіяція програми та перевірка на правильність її роботи через nemiver
\end{DoxyEnumerate}



4.Блок-\/схема

 ~\newline


\begin{DoxyAuthor}{Автор}
Berest D. 
\end{DoxyAuthor}
\begin{DoxyDate}{Дата}
20-\/dec-\/2020 
\end{DoxyDate}
\begin{DoxyVersion}{Версія}
7.\+1.\+5 
\end{DoxyVersion}
